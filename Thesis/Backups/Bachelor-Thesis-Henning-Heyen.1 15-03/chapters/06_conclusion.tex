
\chapter{Conclusion}\label{chapter:conclusion}

To sum up, we find that indeed only strict Nash equilibria survive under no-regret dynamics. If we average trajectories, we also have Nash convergence for two-player zero-sum games with an interior equilibrium. \\

For future research, it would be interesting to study weak pure Nash equilibria in more detail. We observe convergent behavior of no-regret algorithms in games with weak pure Nash equilibria, but they do not converge to Nash but rather to the boundary. As we know, no-regret learning generally converges to the game's set of coarse correlated equilibria. Therefore a possible explanation for that strange convergence behavior could be that the outcome is actually a correlated or coarse correlated equilibrium. \\

Moreover, it might be interesting to make an in-depth comparison between the notions of \textit{attracting} and \textit{stable}. The Intersection game showed that the attracting neighbourhood does not necessarily need to be a subset of the stable neighborhood for some equilibrium. \\

Additionally, one could compare the convergence rates for different no-regret algorithms and tune the step size accordingly. Maybe it might also be worth extending the visualizations to three-player settings, even though it could be a bit tricky. \\

More generally speaking, Nash equilibria as the archetype of solution concepts can be questioned overall as it is so hard to compute and only learnable under strong assumptions. 