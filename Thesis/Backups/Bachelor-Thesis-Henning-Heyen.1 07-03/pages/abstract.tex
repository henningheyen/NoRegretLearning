\chapter{\abstractname}

This thesis addresses the outcome of no-regret dynamics in finite games. Can the outcome be characterized by traditional game theoretic solution concepts like Nash equilibria? The general answer to this question is no. Nevertheless there are some games where Nash convergence under no-regret learning has been observed before. The thesis aims to give a neat and compact overview on sufficient conditions under which no-regret learning converges. These conditions are empirically confirmed by employing two concrete instances of no-regret algorithms on simple two player games. Plots are provided to further give an intuition of the algorithms' behavior. The bottom line is that the sequence of play induced by no-regret algorithms converge to strict Nash equlibria soley and the empirical frequency of play converges to interior Nash equilibria in constant sum games.
