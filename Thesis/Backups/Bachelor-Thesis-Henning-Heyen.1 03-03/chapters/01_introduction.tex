
\chapter{Introduction}\label{chapter:introduction}



\begin{comment}
\begin{itemize}
    \item So far, there is no comprehensive characterization of games that are “learn- able,” but there are some important results. For example, it is well-known that no-regret dynamics converge to a coarse correlated equilibrium in arbitrary fi- nite games [17,26,15,12]. Coarse correlated equilibria (CCE) encompass the set of correlated equilibria (CE) of a finite game. The latter is a nonempty convex polytope which in turn contains the convex hull of the game’s Nash equilibria such that we get NE subset CE subset CCE. (ComputingBNE, p.3)
    \item investigate the question whether NE is learnable via nor regret dynamics \cite{jafari} Introduction
    \item What is the outcome of multi agent learning via no regret algorithms in repeated games? A learning algorithm is said to exhibit no regret iff average payoffs that are achieved by the algorithm exceed tha payoffs that could be achieved by any fixed strategy. in the limit. \cite{jafari} Introduction
    \item best response dynamics like fictitious play does not converge to NE in general  because rational play yields deterministic play. Therefore rational play can not possibly converge to NE in games where no PNE exist. In contrast no regret dynamics are recipes by which to update probabilities that agents assaign to actions could potentially learn MNEs. 
    \item under FoReL only \textit{strict Nash equilibria} survive. 
    
    \item maybe divide in Problem and Contribution
    
    \item There is an unfortunate disconnect between game
    theory and optimization in terms of how objectives are formulated. In optimization,
    the objective is to minimize the incurred cost; in game theory, to maximize one’s rewards. In turn, this creates a clash of terminology when referring to methods such as
    “gradient descent” or “mirror descent” in a maximization setting. To avoid going against
    the traditions of each field, we keep the usual conventions in place (minimization in
    optimization, maximization in game theory), and we rely on the reader to make the
    mental substitution of “descent” to “ascent” when needed (merthabili p.4)
    
    \item we focus on learning via “dual averaging”, a widely used class of no-regret learning schemes where players take small steps along their individual payoff gradients and then “mirror” the output back to their action sets \cite[abstract]{mertikopoulos}
    
    \item The classical problem of finding a Nash equilibrium in multi-agent systems has been a topic of
    prolific research in several areas, including Mathematics, Economics, Algorithmic Game Theory,
    Optimization [Nas50, Sio58, DGP06, NRTV07, Nes05] and more recently Machine Learning in
    the context of Generative Adversarial Networks [GPAM+14, ACB17] and multi-agent reinforcement learning (Solving Zero-Sum Games through Alternating Projections)
    
    \item n this and the following chapter, our aim is to examine the long-run behavior of online
    learning dynamics in a game-theoretic context. Specifically, we seek to address the
    following questions:
    Does no-regret learning lead to rationally admissible states?
    In particular, does it converge to a Nash equilibrium?
    As we discussed towards the end of the previous chapter, the counterexample of
    Viossat and Zapechelnyuk [148] shows that there exist coarse correlated equilibria that
    are supported only on strictly dominated strategies. Therefore, since playing a coarse
    correlated equilibrium leads to no regret [35], the answer to both questions above is, in
    general, a resounding “no”. Especially on the issue of convergence to Nash equilibrium,
    the impossibility result of Hart and Mas-Colell [62] shatters any hope of obtaining an
    unconditionally positive answer when the players’ dynamics are uncoupled – i.e., the
    adjustment of a player’s strategy does not depend explicitly on the payoff functions of the
    other players. All in all, as pointed out by Cesa-Bianchi and Lugosi [35, p. 205] specifying
    the precise interplay between no-regret learning and Nash equilibrium is a “considerably
    more difficult problem” \cite{HDRmertikopoulos} p.33
\end{itemize}
\end{comment}







